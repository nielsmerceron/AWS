\documentclass[a4paper,12pt]{report}
\usepackage[utf8]{inputenc}
\usepackage[T1]{fontenc}
\usepackage[french]{babel}
\usepackage{graphicx}
\usepackage{amsfonts}
\usepackage{float}
\usepackage[export]{adjustbox}
\usepackage{tikz}
\usepackage{algorithm}
\usepackage{algpseudocode}
\usepackage{subfig}
\usepackage[hidelinks]{hyperref}

\date{}
\title{Rapport AWS }
\author{Niels Merceron \\ Pierre Vermeulen \\ Alexis Guigal \\ Manel Azgag \\ \\  \includegraphics[scale=0.20]{logo-UVSQ-2020-RVB.png}}


\begin{document}
	\maketitle
	
	\newpage
	 \tableofcontents
		\chapter{Introduction}
		
			Nous sommes un groupes avec peu de notion de web, donc pour nous familiariser avec les langages et autre technologie nous avons fait le choix de faire un site  simple et plus particulièrement un site de to do list. Il y a beaucoup de ressource en ligne nous décrivant comment faire une to do list et donc nous aiderons tout au cours du dévelopement de notre site pour savoir quel service implémenter de notre site ou non, si c'est a notre portée.
			\\Pour autant, faire ce site relèvera de quelques challenges de conception vu que nous ne sommes pas familier avec beaucoup de technologie pour développer des sites web.
			On peut noter par exemple la gestion du back peut être vite contraignante, ou l'aspect sécurité qui doit être fait de manière la plus propre possible pour ne laisser aucune faille.
			
  
		\chapter{Technologie utilisée}
			
			
		
		\chapter{service implémenté}
			

			
		\chapter{Conclusion}

			
\end{document}
